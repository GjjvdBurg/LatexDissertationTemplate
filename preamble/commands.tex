% command for empty page
\newcommand\blankpage{%
	\null
	\thispagestyle{empty}%
	\newpage}
\newcommand\myemptypage{\afterpage{\blankpage}}

% command for empty line
\newcommand\emptyline{%
	\hspace{1mm}\\
}

% for without using babel
\renewcommand\contentsname{Table of Contents}
% for with using babel
\addto\captionsbritish{%
	\renewcommand{\contentsname}%
	{Table of Contents}%
}
% for with using babel
\addto\captionsenglish{%
	\renewcommand{\contentsname}%
	{Table of Contents}%
}

% Command for a blind footnote
\newcommand\blfootnote[1]{%
	\begingroup
		\renewcommand\thefootnote{}\footnote{\hskip-13pt #1}%
		\addtocounter{footnote}{-1}%
	\endgroup
}

% Renew description label
\renewcommand{\descriptionlabel}[1]{\hspace{\labelsep}{#1}}

% Commands for handling of Dutch last names in references and text
% Usage: In BibTeX use the command \DutchLastName{A}{B}{C} where A determines 
% the sorting of the entry in the bibliography, B determines how the name 
% appears in the text, and C determines how the name appears in the 
% bibliography. Examples: 
%
% author={{\DutchLastName{De}{De}{de}} Leeuw}
%
% would sort this under the "D", appear in the text as "De Leeuw" and in the 
% bibliography as "de Leeuw". The simpler DLN command takes only two 
% arguments, and assumes you will always want to sort on the capitalization, 
% so \DutchLastName{De}{De}{de} = \DLN{De}{de}.
%
% Important: The braces {} around the command in the author name are 
% necessary:
%
%     author={{\DutchLastName{De}{De}{de}} Leeuw}
%             ^                          ^
% otherwise the sorting will add it to the D from "DutchLastName".
%
% From http://tex.stackexchange.com/q/40747/
% using: http://www.tex.ac.uk/faq/FAQ-ltxhash.html
%
\newcommand{\DutchLastNamesBib}{%
	\DeclareRobustCommand{\DutchLastName}[3]{##3}
	\DeclareRobustCommand{\DLN}[2]{##2}%
}

\newcommand{\DutchLastNamesMain}{%
	\DeclareRobustCommand{\DutchLastName}[3]{##2}
	\DeclareRobustCommand{\DLN}[2]{##1}%
}

% Uncomment the next line to use these commands
%\DutchLastNamesMain{}

% Can be used for fancy formatting in the ERIM PhD Series pages
\newcommand{\erimphd}[6]{%
	\vbox{%
		\hangindent=\parindent
		\hangafter=1
		\noindent
		#1,
		\emph{#2},
		#3: #4,
		#5,
		\mbox{\url{#6}}.\\%
	}%
}
